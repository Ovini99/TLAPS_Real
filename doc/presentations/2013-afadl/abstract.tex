\documentclass[a4paper]{article}

\usepackage[utf8]{inputenc}
\usepackage{url}

\title{
  Tutorial on the \tlaplus{} Method and Tools\\
  for Modeling and Verifying Algorithms
}
\author{
  Stephan Merz and Hernán Vanzetto\\
  Inria Nancy Grand-Est and LORIA\\
  MSR-Inria Joint Centre Saclay
}
\date{}

\newcommand{\tlaplus}{{\upshape TLA\textsuperscript{+}}}
\newcommand{\implies}{\ensuremath{\Rightarrow}}


\begin{document}

\maketitle

\section{The \tlaplus{} language and tools}

\tlaplus~\cite{lamport:tlabook,merz:tlabook} is a specification language based
on Zermelo-Fraenkel set theory for modeling data structures and on the
linear-time temporal logic TLA for specifying system executions and their
properties. It is mainly intended for describing discrete systems such as
concurrent and distributed algorithms at a high level of abstraction. TLA
formulas represent specifications as well as properties. Consequently, the
implication $F \implies G$ can alternatively be understood as asserting that $G$
is a property of the system described by $F$, or that $F$ refines a higher-level
specification $G$ of the same system. The underlying refinement notion
corresponds to trace inclusion modulo finite stuttering; it is essentially
analogous to the refinement concept underlying the Event-B
method~\cite{abrial:event-b}, except that full fairness and liveness properties
are available thanks to the expressiveness of temporal logic. In a similar
manner, conjunction of specifications corresponds to parallel composition, and
(existential) quantification over state variables represents hiding.

Developing and analysing \tlaplus{} specifications is supported by the
\tlaplus{} tool set, which includes a parser and semantic analyzer, a pretty
printer, and the finite-state model checker TLC~\cite{yu:model-checking}. In
particular, TLC interprets a fragment of \tlaplus{} that corresponds to
descriptions of state transition systems, and it allows users to verify
invariants and temporal properties (including refinement between specifications)
for finite instances of \tlaplus{} specifications. TLC copes with large state
spaces by using the disk, recent versions also support distributed model
checking on clusters of work stations. The \tlaplus{} tool set is accessible
from the \tlaplus{} Toolbox, an Eclipse-based integrated development
environment.

More recently, constructs for writing proofs have been added to
\tlaplus~\cite{cousineau:tla-proofs}. The proof language lets users write proofs
of \tlaplus{} theorems in a hierarchical style, where proofs are decomposed into
sequences of proof steps. TLAPS, the \tlaplus{} proof
system~\cite{chaudhuri:tlaps}, interprets these proofs and delegates leaf proof
steps to automated proof backends. Currently, backend provers include
Zenon, %~\cite{bonichon:zenon}
a tableau prover for first-order logic and set theory, Isabelle/\tlaplus{}, an
encoding of \tlaplus{} in the Isabelle logical
framework %~\cite{wenzel:isabelle}
that provides automated proof methods, and an interface for SMT
solvers. %~\cite{merz:harnessing}.
Backend provers are expected to provide proof traces that can be checked by
Isabelle/\tlaplus{}, the trusted proof component in TLAPS. Users can work on
different proof steps in any order, and the \tlaplus{} Toolbox indicates the
status of parts of a proof using color coding. It also provides commands that
facilitate writing and reading hierarchical proofs. Finally, TLAPS maintains a
database of previous invocations of backend provers for the current project,
thus supporting status checking and proof maintenance in the presence of
changing specifications and proofs. The integration of TLC and TLAPS in a
unified interface allows users to model check assertions before attempting a
full-fledged proof.

\bigskip

\noindent%
This tutorial will present the \tlaplus{} language and tools through a running
example that progressively introduces different features of the notation, of the
TLC model checker, and of the \tlaplus{} proof system. Depending on the schedule
of AFADL, we propose a duration of two or three hours. It can be presented in
English or French.


\section{About the presenters}

\emph{Stephan Merz} is a senior researcher at Inria Nancy and LORIA. He has
long-standing experience with using the \tlaplus{} formalism and tools, and has
published several papers about the concepts and the logic underlying \tlaplus{}.
He has also presented the formalism and the tool set at different summer schools.
Through a project at the MSR-Inria Joint Centre in Saclay, he has contributed to
the design and development of TLAPS.
%, in particular to Isabelle/\tlaplus{} and the SMT backend.

\medskip

\noindent%
\emph{Hernán Vanzetto} is a PhD student supported by the project at the
MSR-Inria Joint Centre in Saclay for the development of TLAPS. He has in
particular designed and implemented the SMT backend, which has significantly
increased the level of automation of \tlaplus{} proofs.


\bibliographystyle{abbrv}
\bibliography{abstract}

\end{document}
