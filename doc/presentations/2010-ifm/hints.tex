\section{Hints on Using the Prover Effectively}

\begin{frame}
  \frametitle{Control the Size of Formulas}

  \begin{itemize}
  \item \tc{dkblue}{Proof obligations are often large}

    \begin{itemize}
    \o long definitions of actions and invariants
    \o \kw{let} constructions add to complexity when expanded
    \end{itemize}

  \oo \tc{dkblue}{The backend provers are easily overwhelmed by large formulas}

    \begin{itemize}
    \o may work on top-level operators or deeply inside a long formula
    \o even simple proof steps may take an extraordinate amount of time
    \end{itemize}

  \oo \alert{Use local definitions and \HIDE\ them when unnecessary}

    \begin{itemize}
    \o prove facts about a \kw{let}-bound operator, then \HIDE\ it
    \end{itemize}
  \end{itemize}

  \vfill\vfill
\end{frame}

\begin{frame}
  \frametitle{Example: Controlling the Size of Expressions}

  \qquad\begin{tlablock}
    \LEMMA\
    \begin{noj2}
      & \begin{conj}
          x \in SomeVeryBigExpression\\
          y \in AnotherBigExpression
        \end{conj}\\
      \biimplies &
        \begin{conj}
            y \in AnotherBigExpression\\
            x \in SomeVeryBigExpression
        \end{conj}
\pause
        \hspace{1cm}\raisebox{0cm}[0pt][0pt]{\begin{minipage}{3cm}
          \begin{beamercolorbox}[rounded=true,shadow=true]{postit}\footnotesize
            \OBVIOUS\ may take\\forever here
          \end{beamercolorbox}
        \end{minipage}}
    \end{noj2}\\
\pause
    \ps{1}{.}\ \ \ \DEFINE\ S\ \deq\ SomeVeryBigExpression\\
    \ps{1}{.}\ \ \ \DEFINE\ T\ \deq\ AnotherBigExpression\\
    \ps{1}{1.}\ S = SomeVeryBigExpression\\
    \quad\OBVIOUS\\
    \ps{1}{2.}\ T = AnotherBigExpression\\
    \quad\OBVIOUS\\
    \ps{1}{.}\ \ \ \HIDE\ \DEF\ S,\ T\\
    \ps{1}{3.}\ 
      \begin{noj2}
        & x \in S \land y \in T\\
        \biimplies & y \in T \land x \in S
      \end{noj2}\\
    \quad\OBVIOUS\\
    \ps{1}{4.}\ \QED\qquad\BY\ \ps{1}{1},\ \ps{1}{2},\ \ps{1}{3}
  \end{tlablock}
\end{frame}

\begin{frame}
  \frametitle{Avoid Circular Rewrites}

  \begin{itemize}
  \item \tc{dkblue}{Rewriting is often effective for reasoning about equalities}

    \begin{itemize}
    \o idea: replace left-hand side of equality by right-hand side
    \o Isabelle's automatic tactic are based on rewriting
    \o for example, use\ \ \tc{dkgreen}{$x' = x-y \land y'=y$}\ \ to eliminate $x'$ and $y'$    \end{itemize}

\pause

  \oo \tc{dkblue}{Must make sure that rewriting terminates}

    \begin{itemize}
    \o consider\ \ \tc{dkgreen}{$s = f(t) \land t = g(s)$}
    \o Isabelle attempts to reject circular sets of equations
    \o if rejected, proof may get stuck
    \o if not rejected, proof may never terminate
    \end{itemize}

  \oo \alert{Use local definitions, and \HIDE\ them to break loops}
  \end{itemize}
\end{frame}

\begin{frame}
  \frametitle{Circular Rewrites: Example}

  \qquad\begin{tlablock}
    \ps{4}{5.}\ r.name = \str{xyz}\\
    \quad\ps{5}{1.}\ r = [name \mapsto "xyz",\ value \mapsto \alt<1>{r}{\alert{r}}.value]\\
    \quad\quad\BY\ \ps{2}{2}\\
    \quad\ps{5}{2.}\ \QED\\
    \quad\quad\BY\ \ps{5}{1}
  \end{tlablock}

\pause

  \vfill\vfill

  \qquad\alert{The equation in step $\ps{5}{1}$ is circular!}

  \vfill\vfill

\pause

  \qquad\begin{tlablock}
    \ps{4}{5.}\ r.name = \str{xyz}\\
    \quad\ps{5}{}\ \ \ \DEFINE\ rval\ \deq\ r.value\\
    \quad\ps{5}{1.}\ r = [name \mapsto "xyz",\ value \mapsto rval]\\
    \quad\quad\BY\ \ps{2}{2}\\
    \quad\ps{5}{}\ \ \ \HIDE\ \DEF\ rval\\
    \quad\ps{5}{2.}\ \QED\\
    \quad\quad\BY\ \ps{5}{1}
  \end{tlablock}
\end{frame}

\begin{frame}
  \frametitle{Establishing Facts About \CHOOSE}

  \qquad\begin{tlablock}
    \DEFINE\ m\ \deq\ \CHOOSE x \in S : P(x)\\
    \DEFINE\ NoValue\ \deq\ \CHOOSE x : x \notin Value
  \end{tlablock}

  \begin{itemize}
  \oo \tc{dkblue}{How to prove a property $Q(m)$ ?}

    \begin{itemize}
    \o \CHOOSE\ always denotes some value, even if $P(x)$ holds for no $x \in S$
    \end{itemize}

\pause

  \oo \tc{dkblue}{In practice, must establish the two following facts}

    \begin{itemize}
    \o $\E x \in S : P(x)$
    \o $\A x \in S : P(x) \implies Q(x)$
    \o \tlaps\ will then deduce $Q(m)$
    \end{itemize}

\pause

  \oo \tc{dkblue}{Important special case: ``null'' values}

    \begin{itemize}
    \o existence of such a value follows from the library theorem

      \medskip

      \begin{tlablock}
        NoSetContainsEverything\ \deq\ \A S : \E x : x \notin S
      \end{tlablock}
    \end{itemize}
  \end{itemize}
\end{frame}

\begin{frame}
  \frametitle{It's Easier To Prove Something If It's True}

  \begin{itemize}
  \item \tc{dkblue}{All specifications initially contain mistakes}

    \begin{itemize}
    \o errors range from typos to misunderstandings to genuine bugs
    \o formal mathematical definitions are hard to get right
    \end{itemize}

  \oo \tc{dkblue}{\tlaps\ is not good at catching specification errors}

    \begin{itemize}
    \o if you are stuck on a proof, is it you, the prover or the specification?
    \o even with structured proofs, complexity quickly gets out of hand
    \end{itemize}

  \oo \tc{dkblue}{Extensively debug your specifications using \tlc}

    \begin{itemize}
    \o almost all bugs manifest themselves on small instances
    \o run \tlc\ on many properties and inspect the counter-examples
    \end{itemize}
  \end{itemize}
\end{frame}

\begin{frame}
  \frametitle{Focus On The Theorems You Are Interested In}

  \begin{itemize}
  \item \tc{dkblue}{\tlaps\ currently has limited support for theories}

    \begin{itemize}
    \o set theory and functions are fully supported
    \o decision procedure for elementary integer arithmetic
    \o very rudimentary support for sequences
    \end{itemize}

\pause

  \oo \tc{dkblue}{State facts about ``data'' as assumptions}

    \begin{itemize}
    \o do you want to verify an algorithm or basic mathematics?
    \o but --- isn't that dangerous?
    \o it is, but you can validate many assumptions using \tlc
    \o override infinite sets with finite ones in the model, e.g.\ \ \tc{dkgreen}{$Nat\ \deq\ 0..50$}
    \end{itemize}

\pause

  \oo \alert{Theory support will improve slowly and your help is welcome}
  \end{itemize}
\end{frame}

%%% Local Variables: 
%%% mode: latex
%%% TeX-master: "tutorial"
%%% End: 
