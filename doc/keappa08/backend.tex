%%% -*- mode: LaTeX; TeX-master: "main.tex"; -*-

\ifx\master\undefined
\documentclass[a4paper]{easychair}
\usepackage{submission}
\begin{document}
{\let\master\relax%%% -*- mode: LaTeX; TeX-master: "main.tex"; -*-

\ifx\master\undefined
\documentclass[a4paper]{easychair}
\usepackage{submission}
\begin{document}
\fi
%%%% PLEASE DO NOT EDIT ABOVE THIS LINE

\title{A \tlaplus Proof System}

\titlerunning{A \tlaplus Proof System}

% \volumeinfo
% 	{P. Rudnicki, G. Sutcliffe} % editors
% 	{2}                         % number of editors
% 	{KEAPPA 2008}               % event
% 	{1}                         % volume
% 	{1}                         % issue
% 	{1}                         % starting page number


%  Alphabetically by surname
\author{
  Kaustuv Chaudhuri \\
  INRIA \\
  \and
  Damien Doligez \\
  INRIA \\
  \and
  Leslie Lamport \\
  Microsoft Research \\
  \and
  Stephan Merz \\
  INRIA \& Loria
}

\authorrunning{Chaudhuri, Doligez, Lamport, and Merz}

\maketitle

%%%% PLEASE DO NOT EDIT BELOW THIS LINE
\ifx\master\undefined
{\let\master\relax %%% -*- mode: LaTeX; TeX-master: "main.tex"; -*-

\ifx\master\undefined
\documentclass[a4paper]{easychair}
\usepackage{submission}
\begin{document}
{\let\master\relax %%% -*- mode: LaTeX; TeX-master: "main.tex"; -*-

\ifx\master\undefined
\documentclass[a4paper]{easychair}
\usepackage{submission}
\begin{document}
\fi
%%%% PLEASE DO NOT EDIT ABOVE THIS LINE

\title{A \tlaplus Proof System}

\titlerunning{A \tlaplus Proof System}

% \volumeinfo
% 	{P. Rudnicki, G. Sutcliffe} % editors
% 	{2}                         % number of editors
% 	{KEAPPA 2008}               % event
% 	{1}                         % volume
% 	{1}                         % issue
% 	{1}                         % starting page number


%  Alphabetically by surname
\author{
  Kaustuv Chaudhuri \\
  INRIA \\
  \and
  Damien Doligez \\
  INRIA \\
  \and
  Leslie Lamport \\
  Microsoft Research \\
  \and
  Stephan Merz \\
  INRIA \& Loria
}

\authorrunning{Chaudhuri, Doligez, Lamport, and Merz}

\maketitle

%%%% PLEASE DO NOT EDIT BELOW THIS LINE
\ifx\master\undefined
{\let\master\relax \input{rearmatter}}
\end{document}
\fi

% LocalWords:  tex Rudnicki Sutcliffe KEAPPA Kaustuv Chaudhuri INRIA Doligez
% LocalWords:  Merz Loria
}
\fi
%%%% PLEASE DO NOT EDIT ABOVE THIS LINE

\bibliographystyle{plain}
\bibliography{submission}

%%%% PLEASE DO NOT EDIT BELOW THIS LINE
\ifx\master\undefined
\end{document}
\fi

% LocalWords:  tex Paxos
}
\end{document}
\fi

% LocalWords:  tex Rudnicki Sutcliffe KEAPPA Kaustuv Chaudhuri INRIA Doligez
% LocalWords:  Merz Loria
}
\fi
%%%% PLEASE DO NOT EDIT ABOVE THIS LINE


% \llnote{Section 4.2 makes sense only if the reader knows that we currently
%  plan to use Isabelle as the ``ultimate authority'' to check the complete proof.
% This needs to be stated somewhere above---perhaps in the place in Section 3
% that talks about the structural obligation.}
% 
\section{Verifying Proof Obligations}
\label{sec:backend}

% \llnote{I changed the title of this section so it covers Section 4.1 
% (whose title I also changed).}
%\llnote{It would be nice to add an introductory sentence or two here.}
% \ednote{DD}{This is rather redundant with the beginning of section 3
%   and the end of section 4.2}
% \llnote{I removed a bunch of redundant stuff here and added the
% following intro to this section to try to justify its structure.  The
% problem is that Section 4.1 is something of an orphan.  It belongs in
% a section about what the PM does besides generating
% obligations---namely, checking that a proof step is ``meaningful'',
% and translating the obligations to the language of the back-end
% provers.  But since we don't have much to say about this, it gets
% tucked into a section about expanding definitions.  I want to justify
% this by saying that expanding definitions is the only interesting
% thing the PM does other than generating proof obligations.
% I've sort of done that in this intro.  Kaustuv, please check that
% you agree with what I've written---and correct it if you don't.}

Once the \PM\ generates the primitive proof obligations, it must send
them to the back-end provers.  The one non-obvious part of doing this
is deciding whether definitions should be expanded by the \PM\ or by
the prover.  This is discussed in Section~\ref{sec:backend.pm}.
We then describe the state of our two current back-end provers,
Isabelle/\tlaplus\ and Zenon.

% they are sent
% to a back-end prover that will try to verify them with automated
% proof procedures and report success or failure.
% The failure case is reported back to the user who must determine
% whether it failed because the automated procedures are not powerful
% enough (in which case the user will add more details to the proof)
% or because something is wrong with the corresponding leaf proof.
% The structural obligation is also sent to the back-end prover in
% order to provide high assurance of the correctness of the proof.
% 
% In our design, Isabelle plays two roles: on the first hand
% Isabelle/\tlaplus{} (the Isabelle kernel with the \tlaplus axioms)
% is the certifier that provides the ultimate assurance of correctness
% by certifying the proofs provided by the automatic procedures,
% on the other hand Isabelle provides some of the automatic proof procedures
% that are used to verify the proof obligations.

\subsection{Expanding Definitions}
\label{sec:backend.pm}

Expansion of usable definitions cannot be left entirely to the
back-end prover.  The \PM itself must do it for two reasons:
\begin{icom}
\item It must check that the current goal has the right form for a
\TAKE, \WITNESS, or \HAVE\ step to be meaningful, and this can require
expanding definitions.

% \item Using TLA's modal operators such as prime (which we have been
% ignoring), we can define an operator $O$ such that a prover based on
% classical logic could verify an invalid formula if it contains $O$.
% All instances of such operators must be removed by expanding their
% definitions before a proof obligation is sent to the back-end prover.
% (These operators seldom occur in practice.)

\llnote{I rewrote the following.  It could use polishing.}

\item The encoding of TLA+ in a back-end prover would be unsound if a
modal operator like prime ($'$) were represented as an operator.
Hence, encoding a definition like $O(x)\DEF x'$ as a definition in the
prover's logic would be unsound.  All instances of such operators must
be removed by expanding their definitions before a proof obligation is
sent to the back-end prover.  (These operators seldom occur in
practice.)

\end{icom}
%
% \llnote{Kaustuv: Before you object to the preceding sentence, remember
% that \ENABLED\ is not a defined operator.  Since we're not discussing
% \ENABLED\ here, it would be an unnecessary distraction to say that
% the PM must expand~it.}
%
Another reason for the \PM to handle definition expansion is that the
Isabelle/\tlaplus object logic does not provide a direct encoding of
definitions made within proofs.  The \PM therefore now performs all
definition expansions.  We plan to reduce the amount of trusted code
in the \PM\ by lambda-lifting all definitions out of each proof
obligation and using Isabelle meta equality ($\equiv$) with the
\texttt{unfolding} tactic before instructing Isabelle to search for a
proof or to certify a given proof.

% \ednote{SM}{I still don't understand the remark about definitions in
%   Isabelle/\tlaplus. Are you saying that the current implementation of
%   the \PM\ doesn't generate corresponding Isabelle definitions? A
%   constant-level operator definition
%   \[
%     Op(x,y) \DEF exp
%   \]
%   can simply be written like this in Isabelle. ``Lambda-lifting'' is
%   done by the Isabelle toplevel. Level-1 or 2 \tlaplus definitions
%   require a little more care, but we omit them here.}
% 
% \begin{quote}
%   \ednote{KC}{I assume you mean "\And x\ y.\ \mathit{Op}(x, y) \equiv
%     \mathit{exp}", and possibly even "\And (x :: c).\ \And (y :: c).\
%     \mathit{Op}(x, y) \equiv \mathit{exp}" because Isabelle can
%     otherwise infer an overly general type for "\mathit{Op}". Even so,
%     being Pure syntax, such a method only applies to the outermost
%     operator definitions. The only way to use them for inner
%     definitions (\ie, those inside \kwd{LET}-expressions) for the time
%     being is to lift them out.}
% \end{quote}
% 
% \llnote{Kaustuv told me that this kind of definition mechanism, which
%   Isabelle provides, was not implemented as part of Isabelle/\tlaplus\
%   for some reason that I forget.  Stephan seems to be saying that
%   Kaustuv was mistaken, and that he should not be expanding
%   definitions when sending things to Isabelle (except of course for
%   non-Leibniz operators).  Will the two of you please work this out by
%   email or telephone and let us know what the story is.}


\subsection{Isabelle/\tlaplus}
\label{sec:backend.isa}

The core of \tlatwo is being encoded as a new object logic
Isabelle/\tlaplus in the proof assistant
Isabelle~\cite{paulson:isabelle}.
% This encoding serves two distinct
% purposes in the overall project. First, we rely on Isabelle's trusted
% kernel to certify the overall correctness of a proof by checking the
% structural obligation described in Section~\ref{sec:obligations}.
% Second, Isabelle can be used as one of the back-end provers to
% establish leaf steps of a proof.
% 
% \ednote{SM}{I have commented out a remark on the two roles of
%   Isabelle, which Leslie added to the introduction of Sect.~4. If you
%   think it should nevertheless be repeated here, uncomment it.}
% 
One of Isabelle's distinctive features that similar proof assistants
such as Coq~\cite{coq} or HOL~\cite{gordon:hol,harrison:hol} lack is
genericity with respect to different logics. The base system
Isabelle/Pure provides the trusted kernel and a framework in which the
syntax and proof rules of object logics can be defined. We have chosen
to encode \tlatwo as a separate object logic rather than add it on top
of one of the existing logics (such as ZF or HOL). This simplifies the
translation and makes it easier to interpret the error messages when
Isabelle fails to prove obligations. A strongly typed logic such as
HOL would have been unsuitable for representing \tlatwo, which is
untyped. Isabelle/ZF might seem like a natural choice, but differences
between the way it and \tlaplus{} define functions and tuples would
have made the encoding awkward and would have prevented us from
reusing existing theories. Fortunately, the genericity of Isabelle
helped us not only to define the new logic, but also to instantiate the
main automated proof methods, including rewriting, resolution- and
tableau provers, and case-based and inductive reasoning.
Adding support for more specialized reasoning tools such as
proof-producing SAT solvers~\cite{fontaine:automation} or SMT solvers
such as haRVey~\cite{deharbe:decision} will be similarly helped by
existing generic interfaces.

% \ednote{LL}{I did some rewording of the preceding paragraph for
%   readability. I don't think I changed what Stephan was trying to say.
%   However, do COQ and HOL really lack genericity, as I think Stephan
%   claimed?\\[.3em]
% SM: In the sense that they impose a fixed logic (calculus of
% constructions or simply-typed higher order logic). This is not to
% say that these logics are not sufficiently expressive to define
% \tlatwo{} but typically one would have to define new operators for
% the Boolean connectives etc. instead of reusing those that already
% exist. Isabelle/Pure only has implication and quantification, so
% there is not much to redefine. Please tell me if I should make this
% clearer above. (I assumed that the members of the community
% organizing this workshop are familiar with this problem.)}
% 
% \ednote{SM}{ I could elaborate a little on the overall structure of
%   the encoding, in particular the absence of types, but I wonder if
%   this is interesting enough.}
% 
% \llnote{It looks OK to me.  However, assuming that our 10-page limit
%   is not supposed to count the reference list, we do have a reasonable
%   amount of extra space.  But rather than talking about the overall
%   structure, is it possible to talk about one small aspect of the
%   encoding in enough detail to give a flavor of what the whole thing
%   looks like?}
% 
% \ednote{KC}{It is 10 pages including the bibliography and excluding
%   the appendix.}

The current encoding supports only a core subset of \tlatwo, including
propositional and first-order logic, elementary set theory, functions,
and the construction of natural numbers.  Support for arithmetic,
strings, tuples, sequences, and records is now being added; support
for the modal part of \tlatwo{} (variables, priming, and temporal
logic) will be added later.  Nevertheless, the existing fragment can
already be used to test the interaction of the \PM\ with Isabelle and
other back-end provers.  As explained above, Isabelle/\tlaplus is used
both as a back-end prover and to certify proof scripts produced by
other back-end provers such as Zenon.  If it turns out to be
necessary, we will enable the user to invoke 
% 
% LL: I eliminated the following prose, which says nothing that
%     hasn't already been explained
%
%    When the \PM generates a proof obligation, it can pass it to a
%    prover that produces a full proof as an Isar proof script (such as
%    Zenon, described in Section~\ref{sec:backend.zenon} below).
%    Isabelle/\tlaplus processes the proof script and generates a
%    theorem that is certified by the kernel and can be used in
%    subsequent steps.  Alternatively,
% 
one of Isabelle's automated proof methods (such as \texttt{auto} or
\texttt{blast}) by using a dummy theorem, as explained at the end of
Section~\ref{sec:proof-language.lang}.  If the method succeeds, one
again obtains an Isabelle theorem.  Of course,
%
Isabelle/\tlaplus can also be used independently of the \PM, which is
helpful when debugging tactics.
% We plan to validate the interaction by porting some existing case
% studies to Isabelle/\tlaplus, such as the proof of the Disk Paxos
% consensus algorithm~\cite{gafni:disk-paxos} and a verification of
% distributed clock synchronization
% algorithms~\cite{barsotti:verification}.
\llnote{I removed a sentence about porting existing case studies to
Isabelle/\tlaplus\ from the end of the previous paragraph, and I
replaced it with the ``helpful when debugging tactics.''.  I expect
Stephan to rewrite this if he can find something more accurate or
precise to say instead.  The problem with the original is it raised
the question, why not rewrite the (presumably Isar) proofs in \tlatwo\
and run them through the \PM\ rather than directly in
Isabelle/\tlaplus.  I presume we are planning to do this.  Should we
say so?\\[.4em]
%
SM: Indeed, I wanted to say that we plan to port the existing examples
to \tlatwo, using the \PM. The preceding sentence about independent
use of Isabelle/\tlaplus{} indeed obscured the meaning. Anyway, it is
future work and should, if at all, go to the conclusion.}


\subsection{Zenon}
\label{sec:backend.zenon}

Zenon~\cite{bonichon07lpar} is a tableau prover for classical
first-order logic with equality that was initially designed to output
formal proofs checkable by Coq~\cite{coq}.  We have extended Zenon to
output Isar proof scripts for Isabelle/\tlaplus theorems, and the \PM
uses Zenon as a back-end prover, shipping the proofs it produces to
Isabelle for certification.  We have also extended Zenon with direct support
for the \tlatwo logic, including definitions and rules about sets and
functions.  Adding support in the form of rules (instead of axioms) is
necessary because some rules are not expressible as first-order
axioms, notably the rules about the set constructs:
\begin{gather*} \small
  \I[\textit{subsetOf}]{e\in\{x\in S:P\}}{e\in S & P[x:=e]}
  \qquad
  \I[\textit{setOfAll}]{e\in\{d:x\in S\}}{\exists y\in S:e=d[x:=y]}
\end{gather*}
Even for the rules that are expressible as first-order axioms, adding them
as rules makes the proof search procedure much more efficient in practice.
The most important example is extensionality: when set extensionality and
function extensionality are
added as axioms, they apply to every equality deduced by the system,
and pollute the search space with large numbers of irrelevant formulas.
By adding them as rules instead, we can use heuristics to apply them only
in cases where they have some chance of being useful.

Adding support for arithmetic, strings, tuples, sequences, and records
will be done in parallel with the corresponding work on
Isabelle/\tlaplus, to ensure that Zenon will produce proofs that
Isabelle/\tlaplus\ will be able to certify.  Adding temporal logic
will be left to future research.
%
% \llnote{I added to the preceding paragraph an explanation of why
% adding support for those things will be done in parallel with
% the Isabelle work.  Damien, please check that this explanation is
% correct.}
%
We also plan to interface Zenon with Isabelle, so it can be called by a
special Isabelle tactic the same way other tools
%% SM: removed reference to haRVey because the haRVey interface is not
%% currently part of the Isabelle distribution.
are.
This
%
% LL: People are not concerned with how much others have to work
% 
%    represents more implementation work but 
% 
will simplify the \PM\ by giving it a uniform interface to the
back-end provers.  It will also allow using Zenon as an Isabelle tactic
independently of \tlaplus{}.


%%%% PLEASE DO NOT EDIT BELOW THIS LINE
\ifx\master\undefined
{\let\master\relax%%% -*- mode: LaTeX; TeX-master: "main.tex"; -*-

\ifx\master\undefined
\documentclass[a4paper]{easychair}
\usepackage{submission}
\begin{document}
{\let\master\relax %%% -*- mode: LaTeX; TeX-master: "main.tex"; -*-

\ifx\master\undefined
\documentclass[a4paper]{easychair}
\usepackage{submission}
\begin{document}
\fi
%%%% PLEASE DO NOT EDIT ABOVE THIS LINE

\title{A \tlaplus Proof System}

\titlerunning{A \tlaplus Proof System}

% \volumeinfo
% 	{P. Rudnicki, G. Sutcliffe} % editors
% 	{2}                         % number of editors
% 	{KEAPPA 2008}               % event
% 	{1}                         % volume
% 	{1}                         % issue
% 	{1}                         % starting page number


%  Alphabetically by surname
\author{
  Kaustuv Chaudhuri \\
  INRIA \\
  \and
  Damien Doligez \\
  INRIA \\
  \and
  Leslie Lamport \\
  Microsoft Research \\
  \and
  Stephan Merz \\
  INRIA \& Loria
}

\authorrunning{Chaudhuri, Doligez, Lamport, and Merz}

\maketitle

%%%% PLEASE DO NOT EDIT BELOW THIS LINE
\ifx\master\undefined
{\let\master\relax %%% -*- mode: LaTeX; TeX-master: "main.tex"; -*-

\ifx\master\undefined
\documentclass[a4paper]{easychair}
\usepackage{submission}
\begin{document}
{\let\master\relax \input{frontmatter}}
\fi
%%%% PLEASE DO NOT EDIT ABOVE THIS LINE

\bibliographystyle{plain}
\bibliography{submission}

%%%% PLEASE DO NOT EDIT BELOW THIS LINE
\ifx\master\undefined
\end{document}
\fi

% LocalWords:  tex Paxos
}
\end{document}
\fi

% LocalWords:  tex Rudnicki Sutcliffe KEAPPA Kaustuv Chaudhuri INRIA Doligez
% LocalWords:  Merz Loria
}
\fi
%%%% PLEASE DO NOT EDIT ABOVE THIS LINE

\bibliographystyle{plain}
\bibliography{submission}

%%%% PLEASE DO NOT EDIT BELOW THIS LINE
\ifx\master\undefined
\end{document}
\fi

% LocalWords:  tex Paxos
}
\end{document}
\fi

% LocalWords:  tex reasoners sequents Zenon Isar HOL ZF minimizes TLA Paxos SMT
% LocalWords:  Charron Bost Schiper haRVey zenon
