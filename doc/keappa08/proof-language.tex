%%% -*- mode: LaTeX; TeX-master: "main.tex"; -*-

\ifx\master\undefined
\documentclass[a4paper]{easychair}
\usepackage{submission}
\begin{document}
{\let\master\relax%%% -*- mode: LaTeX; TeX-master: "main.tex"; -*-

\ifx\master\undefined
\documentclass[a4paper]{easychair}
\usepackage{submission}
\begin{document}
\fi
%%%% PLEASE DO NOT EDIT ABOVE THIS LINE

\title{A \tlaplus Proof System}

\titlerunning{A \tlaplus Proof System}

% \volumeinfo
% 	{P. Rudnicki, G. Sutcliffe} % editors
% 	{2}                         % number of editors
% 	{KEAPPA 2008}               % event
% 	{1}                         % volume
% 	{1}                         % issue
% 	{1}                         % starting page number


%  Alphabetically by surname
\author{
  Kaustuv Chaudhuri \\
  INRIA \\
  \and
  Damien Doligez \\
  INRIA \\
  \and
  Leslie Lamport \\
  Microsoft Research \\
  \and
  Stephan Merz \\
  INRIA \& Loria
}

\authorrunning{Chaudhuri, Doligez, Lamport, and Merz}

\maketitle

%%%% PLEASE DO NOT EDIT BELOW THIS LINE
\ifx\master\undefined
{\let\master\relax %%% -*- mode: LaTeX; TeX-master: "main.tex"; -*-

\ifx\master\undefined
\documentclass[a4paper]{easychair}
\usepackage{submission}
\begin{document}
{\let\master\relax %%% -*- mode: LaTeX; TeX-master: "main.tex"; -*-

\ifx\master\undefined
\documentclass[a4paper]{easychair}
\usepackage{submission}
\begin{document}
\fi
%%%% PLEASE DO NOT EDIT ABOVE THIS LINE

\title{A \tlaplus Proof System}

\titlerunning{A \tlaplus Proof System}

% \volumeinfo
% 	{P. Rudnicki, G. Sutcliffe} % editors
% 	{2}                         % number of editors
% 	{KEAPPA 2008}               % event
% 	{1}                         % volume
% 	{1}                         % issue
% 	{1}                         % starting page number


%  Alphabetically by surname
\author{
  Kaustuv Chaudhuri \\
  INRIA \\
  \and
  Damien Doligez \\
  INRIA \\
  \and
  Leslie Lamport \\
  Microsoft Research \\
  \and
  Stephan Merz \\
  INRIA \& Loria
}

\authorrunning{Chaudhuri, Doligez, Lamport, and Merz}

\maketitle

%%%% PLEASE DO NOT EDIT BELOW THIS LINE
\ifx\master\undefined
{\let\master\relax \input{rearmatter}}
\end{document}
\fi

% LocalWords:  tex Rudnicki Sutcliffe KEAPPA Kaustuv Chaudhuri INRIA Doligez
% LocalWords:  Merz Loria
}
\fi
%%%% PLEASE DO NOT EDIT ABOVE THIS LINE

\bibliographystyle{plain}
\bibliography{submission}

%%%% PLEASE DO NOT EDIT BELOW THIS LINE
\ifx\master\undefined
\end{document}
\fi

% LocalWords:  tex Paxos
}
\end{document}
\fi

% LocalWords:  tex Rudnicki Sutcliffe KEAPPA Kaustuv Chaudhuri INRIA Doligez
% LocalWords:  Merz Loria
}
\fi
%%%% PLEASE DO NOT EDIT ABOVE THIS LINE

\section{\tlaplus and its Proof Language}
\label{sec:proof-language}

\subsection{TLA}
\label{sec:proof-language.tla} 

The \tlaplus\ language is based on the Temporal Logic of Actions
(TLA)~\cite{lamport:newtla}, a linear-time temporal logic. The rigid
variables of TLA are called \emph{constants} and the flexible
variables are called simply \emph{variables}.  TLA assumes an
underlying ordinary (non-modal) logic for constructing expressions.
Operators of that logic are called \emph{constant} operators.  A
\emph{state function} is an expression built from constant operators
and TLA constants and variables.  The elementary (non-temporal)
formulas of TLA are \textit{actions}, which are formulas built with
constant operators, constants, variables, and expressions of the form
$f'$, where $f$ is a state function.  (TLA also has an \ENABLED
operator that is used in reasoning about liveness, but we ignore it
for brevity.)  An action is interpreted as a predicate on pairs of
states that describes a set of possible state transitions, where state
functions refer to the starting state and primed state functions refer
to the ending state.  Because priming distributes over constant
operators and because $c'$ is equal to $c$ for any constant $c$, an
action can be reduced to a formula built from constant operators,
constants, variables, and primed variables.

TLA is practical for describing systems because all the complexity of
a specification is in the action formulas.  Temporal operators are
essentially used only to assert liveness properties, including
fairness of system actions.  Most of the work in a TLA proof is in
proving action formulas; temporal reasoning occurs only in proving
liveness properties and is limited to propositional temporal logic and
to applying a handful of proof rules whose main premises are action
formulas.  Because temporal reasoning is such a small part of TLA
proofs, we have deferred its implementation.  The \PM\ now handles
only action formulas.  We have enough experience mechanizing TLA's
temporal reasoning~\cite{engberg:mechanical} to be fairly confident
that it will not be hard to extend the \PM to support it.

A formula built from constant operators, constants, variables, and
primed variables is valid iff it is a valid formula of the underlying
logic when constants, variables, and primed variables are treated as
distinct variables of the logic---that is, if $v$ and $v'$ are
considered to be two distinct variables of the underlying logic, for
any TLA variable $v$.  Since any action formula is reducible to such a
formula, action reasoning is immediately reducible to reasoning in the
underlying logic.  We therefore ignore variables and priming here and
consider only constant formulas.


\subsection{\tlaplus}

The \tlaplus\ language adds the following to the TLA logic:
\begin{icom}
\item An underlying logic that is essentially ZFC set theory plus
  classical untyped first-order logic with Hilbert's
  $\varepsilon$~\cite{leisenring:mathematical-logic}.
% LL: I deleted
% 
%    which is written \CHOOSE\ in \tlaplus.  
% 
% to save a line and avoid unnecessarily introducing a name.
% 
The major difference between this underlying
  logic and traditional ZFC is that functions are defined axiomatically 
  rather than being represented as sets of ordered pairs.

\item A mechanism for defining operators, where a user-defined
operator is essentially a macro that is expanded syntactically.
(\tlaplus\ permits recursive function definitions, but they are
translated to ordinary definitions using Hilbert's $\varepsilon$.)

%% LL: changed ``A module language'' to ``Modules''
\item Modules, where one module can import definitions
  and theorems from other modules.  A module is parameterized by its
  declared variables and constants, and it may be instantiated in another
  module by substituting expressions for its parameters. The
  combination of substitution and the \ENABLED\ operator introduces
  some complications, but space limitations prevent us from discussing
  them, so we largely ignore modules in this paper.
\end{icom}
\tlaplus\ has been extensively documented~\cite{lamport03tla}.  Since
we are concerned only with reasoning about its underlying logic, which
is a very familiar one, we do not bother to describe \tlaplus\ in any
detail.  All of its nonstandard notation that appears in our examples is
explained.

\subsection{The Proof Language}
\label{sec:proof-language.lang}

The major new feature of \tlatwo\ is its proof language.  (For reasons
having nothing to do with proofs, \tlatwo\ also introduces recursive
operator definitions, which we ignore here for brevity.)  We describe
the basic proof language, omitting a few constructs
%%
% LL: I don't think we omitted any ``uninteresting'' constructs,
%     so I deleted:
% 
% 
%    for simplicity a few constructs that are either uninteresting or
%    
% 
that concern aspects such as module instantiation that we are not
discussing.  \tlatwo\ also adds constructs for naming subexpressions
of a definition or theorem, which is important in practice for writing
proofs but is orthogonal to the concerns of this paper.

The goal of the language is to make proofs easy to read and write for
someone with no knowledge of how the proofs are being checked.  This
leads to a mostly declarative language, built around the uses and
proofs of assertions rather than around the application of
proof-search tactics.  It is therefore more akin to
Isabelle/Isar~\cite{isar} than to more operational interactive
languages such as Coq's Vernacular~\cite{coq}.
Nevertheless, the proof language does include a few operational
constructs that can eliminate the repetition of common idioms, albeit
with some loss of perspicuity.

% at the cost of some loss of perspicuity.


At any point in a \tlaplus\ proof, there is a current obligation that
is to be proved.  The obligation contains a \emph{context} of
\emph{known} facts, definitions, and declarations, and a
\emph{goal}.
%%
%% SM: slight rewording of the following
%%
% which is a proposition that the obligation claims to be entailed by
% the context.
The obligation claims that the goal is logically entailed by the context.
Some of the facts and definitions in the
context are marked as \emph{usable} for reasoning, while the remaining
facts and definitions are \textit{hidden}.
%
%% LL: We'd better say more about this below, so it's redundant
%% here:
% 
% and may not be used in the
% proof unless explicitly made usable with a \USE declaration (see
% below).

\ednote{SM}{I don't think we are quite consistent about what is an
  assertion. This presentation suggests that an assertion is not just
  an \ASSUME{} \PROOF{}, but also ``contains'' the context in which
  that \ASSUME{} \PROOF{} appears. I think that the term ``assertion''
  is used differently towards the end of this section, although the
  description here matches what we say in Section 3.}

Proofs are structured hierarchically.
%
%% LL: This is deferred until the first use of PROOF.
% and optionally begin with the token \PROOF. 
%
% At the lowest levels of the hierarchy are the
% \textit{leaf proofs}.  
The leaf (lowest-level) proof \OBVIOUS\ asserts that the
current goal follows easily from the usable facts and definitions.
The leaf proof
\begin{gather*}
  \BY\ e_{1},\ldots, e_{m} \ \DEFS\ o_{1},\ldots, o_{n}
\end{gather*}
asserts that the current goal follows easily from the usable facts and
definitions together with (i)~the facts $e_{i}$ that must themselves
follow easily from the context and (ii)~the known definitions of
$o_{j}$.  Whether a goal follows easily from definitions and facts
depends on who is trying to prove it.  The \PM\ generates proof
obligations for each leaf proof, so in practice ``follows easily''
means that a back-end prover can prove them.

A non-leaf proof is a sequence of \textit{steps}, each consisting of a
begin-step token and a proof construct.  For some constructs
(including a simple assertion of a proposition) the step takes a
subproof, which may be omitted.  The final step in the sequence simply
asserts the current goal, which is represented by the token \QED.
%
A begin-step token is either a \emph{level token} of the form \s{n} or
a \emph{label} of the form \s{n}"l", where $n$ is a level number that
is the same for all steps of this non-leaf proof, and "l" is an
arbitrary name.  The hierarchical structure is deduced from the level
numbers of the begin-step tokens, a higher level number beginning a
subproof.

%% LL: I changed 
%   A step that makes declarations or definitions or that changes the form
%   of the current goal does not require a proof,
%   but one that makes an assertion is followed by its proof 
%
% because (a) SUFFICES changes the goal and requires a proof and 
% (b) PICK makes a definition and requires a proof
% (c) WITNESS e \in S makes an assertion but does not take a proof.
%  Also assertions whose proof is omitted are not followed by their proof--
%  at least not with the reader's current understanding of proof.
%
Some steps make declarations or definitions or change the current
goal and do not require a proof.  Other steps make assertions that
become the current goals for their proofs.  An omitted proof (or one
consisting of the token \OMITTED) is considered to be a leaf proof
that instructs the assertion to be accepted as true.  Of course, the
proof is then incomplete and cannot be certified.  Following a step
that makes an assertion (and the step's proof), until the end of the
current proof (after the \QED\ step), the contexts contain that
assertion in their sets of known facts.  The assertion is marked
usable iff the begin-step token is a level token; otherwise it can be
referred to by its label in a \BY\ proof.

The hierarchical structure of proofs not only aids in reading finished
proof but is also quite useful in incrementally writing proofs.  The
steps of a non-leaf proof are first written with all proofs but that
of the \QED\ step omitted.  After checking the proof of the \QED
step, the proofs omitted for other steps in this or earlier levels
are written in any order.  When writing the proof, one may discover
facts that are needed in the proofs of multiple steps.
% 
% LL: I modified the following text because we say in conclusion
%     that adding lemmas is bad. 
%
Such a fact is then added to the proof as an early step, or added at a
higher level.  It can also be removed from the proof of the theorem
and proved separately as a lemma.  However, the hierarchical proof
language encourages facts relevant only for a particular proof to be
kept within the proof, making the proof's structure easier to see and
simplifying maintenance of the proof.  For correctness proofs of
systems, the first few levels of the hierarchy are generally
determined by the structure of the formula to be proved---for example,
the proof that a formula implies a conjunction usually consists of steps
asserting that it implies each conjunct.

As an example, we incrementally construct a hierarchical proof of
Cantor's theorem, which states that there is no surjective function
from a set to its powerset. It is written in \tlaplus as:
%
\begin{quote} \small
  \begin{tabbing}
    \THEOREM\ "\forall S : \forall f \in [S -> \SUBSET\ S] : 
        \exists A \in \SUBSET\ S : \forall x \in S : f[x] \neq A"
  \end{tabbing}
\end{quote}
%
%% SM: cosmetic change of the following that heeds Leslie's advice below
%% (I think it used to be more or less like that before, why did it
%% change?)
%%
%where "f[x]" is the application of the function "f" to "x",
where function application is written using square brackets,
"\SUBSET\ S" is the powerset of "S", and "[S -> T]" is the set of
functions from $S$ to $T$.
% \llnote{One should not write two mathematical operators like
% $x$ and $\SUBSET$ separated only by punctuation, since that is hard to
% read.  The sentence above needs to be rewritten.}

The statement of the theorem is the current goal for 
its top-level proof. A goal of the form $\forall v:e$ is
proved by introducing a generic constant and proving the formula
obtained by 
substituting it for the bound identifier. We express this as follows,
using the \ASSUME/\PROVE construct of \tlatwo:
% \llnote{\ASSUME/\PROVE should be introduced earlier, where we mention
% assertions---probably back where we introduce "||-".}

\begin{quote} \small
  \begin{tabbing}
    \THEOREM\ "\forall S : \forall f \in [S -> \SUBSET\ S] : 
                \exists A \in \SUBSET\ S : \forall x \in S : f[x] \neq A" \\
%%    \PROOF \\
    \LSP \= \s11.\ \= \ASSUME \= "\NEW\ S", \\
         \>        \>         \> "\NEW\ f \in [S -> \SUBSET\ S]"\\
         \>        \> \PROVE "\exists A \in \SUBSET\ S : \forall x \in S : f[x] \neq A" \\
         \> \s12.  \> \QED \BY \s11
  \end{tabbing}
\end{quote}
%
%
Step \s11 asserts that for any constants "S" and "f" with "f \in [S
-> \SUBSET\ S]", the proposition to the right of the \PROVE is true.
More precisely, the current context for the proof of \s11 (which we
have yet to write) contains the declarations of $S$ and $f$ and the
usable fact "f \in [S -> \SUBSET\ S]", and the \PROVE\ assertion is
its goal.  The \QED step states that the original goal (the theorem)
follows from the assertion in step~\s11.

We tell the \PM to check this (incomplete) proof, which it does by
having the back-end provers verify the proof obligation for the \QED
step.  The verification succeeds, and we now continue by writing the
proof of \s11.  (Had the verification failed because \s11 did not
imply the current goal, we would have caught the error before
attempting to prove \s11, which we expect to be harder to do.)

\llnote{Check minor edit below.}

We optimistically start with the proof \OBVIOUS, but it is too hard
for the back-end to prove, and the \PM reports a timeout.  Often this
means that a necessary fact or definition in the context is hidden and
we merely have to make it usable with a \USE step or a \BY proof.  In
this case we have no such hidden assumptions, so we must refine the
goal into simpler goals with a non-leaf proof.  We let this proof have
level 2 (we can use any level greater than 1).  Since the goal itself
is existentially quantified, we must supply a witness.  In this case,
the witness is the classic diagonal set, which we call~"T".
%
\begin{quote} \small
  \begin{tabbing}
    \THEOREM\ "\forall S : \forall f \in [S -> \SUBSET\ S] : \exists A \in \SUBSET\ S : \forall x \in S : f[x] \neq A" \kill
    \PROOF \kill
    \LSP \= \s11.\ \= \ASSUME \= "\NEW\ S", \\
         \>        \>         \> "\NEW\ f \in [S -> \SUBSET\ S]" \\
         \>        \> \PROVE "\exists A \in \SUBSET\ S : \forall x \in S : f[x] \neq A" \\
%         \>        \> \PROOF \\
         \>   \hspace{1em}     \= \s21.\ \= \DEFINE "T \DEF \{z \in S : z \notin f[z]\}" \\
         \>        \> \s22.  \> "\forall x \in S : f[x] \neq T" \\
         \>        \> \s23.  \> \QED \BY \s22
  \end{tabbing}
\end{quote}
Because definitions made within a proof are usable by default, the
definition of $T$ is usable in the proofs of \s22\ and \s23.  Once
again, the proof of the \QED\ step is automatically verified, so all
that remains is to prove \s22.  (The \DEFINE\ step requires no proof.)

% \llnote{I added to the paragraph above an explanation of why the
%   definition of $T$ is usable.  Stephan asks if it wouldn't it be more
%   consistent to give the definition a level token.  In a comment that
%   got deleted along the way, I proposed changing the language to make
%   labeled definitions unusable and allow the user to write $\BY\
%   \DEFS\ \s21$ as well as $\BY\ \DEFS\ T$.  However, we had already
%   decided that the user is most likely to want local definitions to be
%   \USE{}d by default, and many users might prefer to give step numbers
%   to all their steps.  So, we decided to defer a decision on this
%   point.  In the conclusion, I added this as the kind of fine tuning
%   we are likely to make to the language.}
% 
% \ednote{SM}{Fine with me. There must be a way to hide definitions
%   because they can get in the way. Your suggestion to make labeled
%   definitions unusable looks clean and consistent to me, but in its
%   absence we can at least write HIDE DEF.}
% 
% \ednote{KC}{In my opinion, the presence or absence of a label should
%   not be the deciding factor on whether the contents of the step, be
%   they definitions or facts, are usable or hidden. Instead, the label
%   should determine whether the contents of the step are namable by the
%   step token or not. Since definitions are not namable, it does not
%   matter whether we number "\DEFINE" steps or not. (Subexpressions of
%   the body of the definition are reachable via the defined operator
%   name.)}

%
% \llnote{This example suggests that we should modify the current language
% definition so that definitions in a labeled step are not automatically
% usable.  There's no reason why they need to be, and it would make it
% clear that the \QED\ step's proof doesn't depend on the definition.
% We could allow BY DEF \s21\ to include all the definitions made
% in the step.}

The system 
%% LL: changed ``will accept'' to ``accepts'' because it does so right now.
accepts \OBVIOUS\ as the proof of \s22 because the only
difficulty in the proof of \s11 is finding the
witness. However, suppose we want to add another level of proof for the benefit
of a human reader.  The universal quantification is proved as above,
by introducing a fresh constant:

\begin{quote} \small
  \begin{tabbing}
    \THEOREM\ "\forall S : \forall f \in [S -> \SUBSET\ S] : \exists A \in \SUBSET\ S : \forall x \in S : f[x] \neq A" \kill
    \PROOF \kill
    \LSP \= \s11.\ \= \ASSUME \= "\NEW\ S", \kill
         \>        \>         \> "\NEW\ f \in [S -> \SUBSET\ S]" \kill
         \>        \> \PROVE "\exists A \in \SUBSET\ S : \forall x \in S : f[x] \neq A" \kill
         \>        \> \PROOF \kill
         \>   \hspace{1em} \= \s21.\ \= \DEFINE "T == \{z \in S : z \notin f[z]\}" \kill
         \>        \> \s22.  \> "\forall x \in S : f[x] \neq T" \\
         \>        \> \hspace{1em} \= \s31.\ \= \ASSUME "\NEW\ x \in S" \PROVE "f[x] \neq T" \\
         \>        \>        \> \s32.\ \> \QED \BY \s31
  \end{tabbing}
\end{quote}
%
% Naturally, the system verifies the proof of the \QED\ step.  (Remember
% that it could verify \s22\ by itself.)
%     the QED step has nothing to do with \s22  -- KC
\ednote{KC}{Should be able to trim the text for this last level below}
Naturally, the \QED step is verified.  Although the system accepts
\OBVIOUS\ as the proof of \s31 (remember that it could verify \s22\ by
itself), we can provide more detail with yet another level
of proof.  We write this proof the way it would seem natural to a
person---by breaking it into two cases:
%
\begin{quote} \small
  \begin{tabbing}
    \THEOREM\ "\forall S : \forall f \in [S -> \SUBSET\ S] : \exists A \in \SUBSET\ S : \forall x \in S : f[x] \neq A" \kill
    \PROOF \kill
    \LSP \= \s11.\ \= \ASSUME \= "\NEW\ S", \kill
         \>        \>         \> "\NEW\ f \in [S -> \SUBSET\ S]" \kill
         \>        \> \PROVE "\exists A \in \SUBSET\ S : \forall x \in S : f[x] \neq A" \kill
         \>        \> \PROOF \kill
         \> \hspace{1em} \= \s21.\ \= \DEFINE "T == \{z \in S : z \notin f[z]\}" \kill
         \>        \> \s22.  \> "\forall x \in S : f[x] \neq T" \kill
         \>        \>  \hspace{1em} \= \s31.\ \= \ASSUME "\NEW\ x \in S" \PROVE "f[x] \neq T" \\
         \>        \>        \> \hspace{1em} \= \s41.\ \= \CASE "x \in T" \\
         \>        \>        \>        \> \s42.\ \> \CASE "x \notin T" \\
         \>        \>        \>        \> \s43.\ \> \QED \BY \s41, \s42
  \end{tabbing}
\end{quote}
The (omitted) proof of the \CASE\ statement \s41\ has as its goal
"f[x]\neq T" and has the additional usable fact $x\in T$ in its context.

\llnote{Minor edit below.}

We continue refining the proof in this way, stopping 
with an \OBVIOUS or \BY proof when a goal is
obvious enough for the back-end prover or for a human reader,
depending on who the proof is being written for. 
A \BY\ statement can guide the prover or the human reader
by listing helpful obvious consequences of known facts.  For example,
the proof of \s41\ might be "\BY\ x \notin f[x]".
%
The complete proof appears in Appendix~\ref{apx:cantor}.

This example illustrates how the proof language supports a
hierarchical, non-linear, and incremental development of proofs.  The
proof writer can work on the most problematic unproved steps first,
leaving the easier ones for later.  Finding that a step cannot be
proved (for example, because it is invalid) may require changing other
steps, making proofs of those other steps wasted effort.  We intend to
provide an interface to the \PM\ that will make it easy for the user
to indicate which proofs should be checked and will avoid
unnecessarily rechecking proofs.

The example also shows how already-proved facts are generally not made
usable, but are invoked explicitly in \BY\ proofs.  Global definitions
are also hidden by default and the user must explicitly make them
usable.  This makes proofs easier to read by telling the reader what
facts and definitions are being used to prove each step.  It also
helps constrain the search space for an automated back-end prover,
leading to more efficient verification.  Facts and definitions can be
switched between usable and hidden by \USE\ and \HIDE\ steps, which
have the same syntax as \BY. As noted above, omitting the label from a
step's starting token (for example, writing \s4 instead of \s42) makes
the fact it asserts usable.  This might be done for compactness at
the lowest levels of a proof.

% The Isar proof language~\cite{isar} is similarly explicit about
% usable assumptions, but the facility is local to a proof and does
% not persist into sub-proofs, causing repetition in sibling
% sub-proofs. In the \tlatwo proof language, the visibility of an
% assumption persists until hidden, so one \USE declaration at a
% suitable level is enough. (The assumption can be selectively hidden
% at lower levels if needed for performance reasons.)

% \llnote{The preceding stuff about Isar needs to be moved to the conclusion.}


The example also indicates how the current proof obligation at every
step of the proof is clear, having been written explicitly in a parent
assertion.  This clear structure comes at the cost of introducing many
levels of proof, which can be inconvenient.  One way of avoiding these
extra levels is by using an assertion of the form "\SUFFICES\ A",
which asserts that proving $A$ proves the current goal, and makes $A$
the new current goal in subsequent steps.  In our example proof, one
level in the proof of step \s22\ can be eliminated by writing the
proof as:
\begin{quote} \small
  \begin{tabbing}
    \THEOREM\ "\forall S : \forall f \in [S -> \SUBSET\ S] : \exists A \in \SUBSET\ S : \forall x \in S : f[x] \neq A" \kill
    \PROOF \kill
    \LSP \= \s11.\ \= \ASSUME \= "\NEW\ S", \kill
         \>        \>         \> "\NEW\ f \in [S -> \SUBSET\ S]" \kill
         \>        \> \PROVE "\exists A \in \SUBSET\ S : \forall x \in S : f[x] \neq A" \kill
         \>        \> \PROOF \kill
         \> \hspace{1em} \= \s21.\ \= \DEFINE "T == \{z \in S : z \notin f[z]\}" \kill
         \>        \> \s22.  \> "\forall x \in S : f[x] \neq T" \\
         \>        \>  \hspace{1em} \= \s31.\ \= \SUFFICES \ASSUME "\NEW\ x \in S" \PROVE "f[x] \neq T" \\
         \>        \>        \> \hspace{1em} \PROOF \OBVIOUS \\
         \>        \>        \> \s32.\ \> \CASE "x \in T" \\
         \>        \>        \> \s33.\ \> \CASE "x \notin T" \\
         \>        \>        \> \s34.\ \> \QED \BY \s32, \s33
  \end{tabbing}
\end{quote}
where the proofs of the \CASE\ steps are the same as before.  The
\SUFFICES\ statement changes the current goal of the level-3 proof to
$f[x]\neq T$ after adding a declaration of "x" and the usable fact "x
\in S" to the context. This way of proving a universally quantified
formula is sufficiently common that \tlatwo\ provides a \TAKE\
construct that allows the \SUFFICES\ assertion \s31 and its \OBVIOUS
proof to be written \mbox{\,$\TAKE\ x \in S$\,}. 

There is a similar construct, "\WITNESS\ f \in S" for proving an
existentially quantified goal $\exists x\in S: e$, which changes the
goal to "e[x := f]".
% the := notation for capture-avoiding substitution is universal
% enough, especially in the community reading this paper
For implicational goals "e => f", the construct $\HAVE\ e$ changes the
goal to $f$.  No other constructs in the \tlatwo proof language change
the form of the current goal. We advise that these constructs be used
only at the lowest levels of the proof, since the new goal they create
must be derived %% SM: commented out "algebraically"
instead of being available textually in
a parent assertion.  (As a check and an aid to the reader, one can at
any point insert a redundant \SUFFICES\ step that simply asserts the
current goal.)

The final \tlatwo\ proof construct is $\PICK\ x : e$, which introduces
a new symbol $x$ that satisfies $e$.  The goal of the proof of this
\PICK step is $\exists x : e$, and it changes the context of
subsequent steps by adding a declaration of "x" and the fact "e". 
%
A more formal summary of the language appears in
Appendix~\ref{apx}.

The semantics of a \tlatwo\ proof is independent of any back-end
prover. Different provers will have different notions of what
``follows easily'', so an \OBVIOUS\ proof may be verified by one
prover and not another.  In practice, many provers such as Isabelle
must be directed to use decision procedures or special tactics to
prove some assertions.  For this purpose, special standard modules
will contain dummy theorems for giving directives to the
\PM.  Using such a theorem (with a \USE\ step or \BY\ proof) will
% LL: What generated proof obligation?  And why filter something
%     rather than not putting it into proof obligations in the 
%     first place?
cause the \PM\ not to use it as a fact, 
but instead to generate special directives for back-end
provers.  It could even cause the \PM\ to use a different back-end
prover.  (The dummy theorem will assert a true fact that suggests the
purpose of the directive.)
For instance, using the theorem \emph{Arithmetic}
%% LL: Engineers have never heard of Peano Arithmetic, and we don't
%%     have so much space that we can squander it on such an example.
%%   \textit{PeanoArithmetic} might assert the existence of a set of
%%   natural numbers satisfying the axioms of Peano Arithmetic; a "\BY\
%%   \mathit{PeanoArithmetic}" proof 
%
might be interpreted as an instruction to use a decision procedure for
integers.

\llnote{I replaced ``natural numbers'' by ``integers'' above, which seems
to me what arithmetic reasoning is more likely to mean.}

We hope that almost all uses of this feature will leave the \tlatwo
proof independent of the back-end prover(s).  The proof will not have
to be changed if the \PM\ is reconfigured to replace one decision
procedure with a different one.

%%%% PLEASE DO NOT EDIT BELOW THIS LINE
\ifx\master\undefined
{\let\master\relax%%% -*- mode: LaTeX; TeX-master: "main.tex"; -*-

\ifx\master\undefined
\documentclass[a4paper]{easychair}
\usepackage{submission}
\begin{document}
{\let\master\relax %%% -*- mode: LaTeX; TeX-master: "main.tex"; -*-

\ifx\master\undefined
\documentclass[a4paper]{easychair}
\usepackage{submission}
\begin{document}
\fi
%%%% PLEASE DO NOT EDIT ABOVE THIS LINE

\title{A \tlaplus Proof System}

\titlerunning{A \tlaplus Proof System}

% \volumeinfo
% 	{P. Rudnicki, G. Sutcliffe} % editors
% 	{2}                         % number of editors
% 	{KEAPPA 2008}               % event
% 	{1}                         % volume
% 	{1}                         % issue
% 	{1}                         % starting page number


%  Alphabetically by surname
\author{
  Kaustuv Chaudhuri \\
  INRIA \\
  \and
  Damien Doligez \\
  INRIA \\
  \and
  Leslie Lamport \\
  Microsoft Research \\
  \and
  Stephan Merz \\
  INRIA \& Loria
}

\authorrunning{Chaudhuri, Doligez, Lamport, and Merz}

\maketitle

%%%% PLEASE DO NOT EDIT BELOW THIS LINE
\ifx\master\undefined
{\let\master\relax %%% -*- mode: LaTeX; TeX-master: "main.tex"; -*-

\ifx\master\undefined
\documentclass[a4paper]{easychair}
\usepackage{submission}
\begin{document}
{\let\master\relax \input{frontmatter}}
\fi
%%%% PLEASE DO NOT EDIT ABOVE THIS LINE

\bibliographystyle{plain}
\bibliography{submission}

%%%% PLEASE DO NOT EDIT BELOW THIS LINE
\ifx\master\undefined
\end{document}
\fi

% LocalWords:  tex Paxos
}
\end{document}
\fi

% LocalWords:  tex Rudnicki Sutcliffe KEAPPA Kaustuv Chaudhuri INRIA Doligez
% LocalWords:  Merz Loria
}
\fi
%%%% PLEASE DO NOT EDIT ABOVE THIS LINE

\bibliographystyle{plain}
\bibliography{submission}

%%%% PLEASE DO NOT EDIT BELOW THIS LINE
\ifx\master\undefined
\end{document}
\fi

% LocalWords:  tex Paxos
}
\end{document}
\fi
